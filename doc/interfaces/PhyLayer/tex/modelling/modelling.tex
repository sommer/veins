\newcommand{\h}[1]{\underline{#1}}
\newcommand{\bp}{BasePhyLayer}
\newcommand{\bm}{BaseMacLayer}


\section{modelling}

\subsection{overview}

Here we present the design- and interface details of the OMNeT-module \h{\bp} and go step-by-step through the requirement specification:

\begin{enumerate}
 \item internal class diagram of \h{\bp} and relation (pointers, references and OMNeT-gates) to \h{\bm}
 \item interface description for all involved \textsf{C++}-classes
 \item flow charts for reception of MacPacket from upper layer and AirFrame from the channel
 \item detailed flow chart for the receiving process
\end{enumerate}

\subsection{classgraph}

We start with the classgraph for the OMNeT-module \h{\bp} that shows its
\textsf{C++}-classes, relations to other OMNeT-modules (especially \h{\bm})
and OMNeT-messages sent between them.

\begin{figure}[h]
 \centering
 \includegraphics[width=400pt]{modelling/class_diagram.png}
 \caption{class graph}
 \label{fig: classgraph}
\end{figure}

\subsection{the \h{\bp} interface}

In this section we focus on how one is able to communicate with the \h{\bp}, i.e. 
especially the \h{\bm} which is connected via an OMNeT-channel, OMNeT-controlchannel and holds a reference to \h{\bp}.
\h{\bm} can obtain information about the channelstate, i.e. idle (boolean) or RSSI by 
a simple method call. Further it is able to set current mode in \h{\bp} via method call.

\begin{figure}[h]
 \centering
 \includegraphics[width=340pt]{modelling/BasePhyLayer_members.png}
 \caption{BasePhyLayer interface}
 \label{fig: BasePhyLayer interface}
\end{figure}



\subsection{AnalogueModel and Signal}


The Signal is designed one-dimensional (value-by-time) by default. The owner is able
to add and request values at a specific time point.
The Method getTimeIterator() return an appropriate SignalIterator.
\begin{quote}
\emph{BEWARE: Anyone who subclasses Signal should make shure to have a properly
working SignalIterator (subclassed) for it.}
\end{quote}

\begin{figure}[h]
 \centering
 \includegraphics[width=340pt]{modelling/AnalogueModel_members.png}
 \caption{analogue model interface}
 \label{fig: analogue model interface}
\end{figure}

The AnalogueModel offers functionality to filter a referenced signal in a specified
interval and at a specific point in time. Therefore an appropriate SignalIterator is needed!

\begin{figure}[h]
 \centering
 \includegraphics[width=300pt]{modelling/apply_analogue_modells_detail.png}
 \caption{application of analogue models}
 \label{fig: application analogue models}
\end{figure}


\subsection{AirFrame}


\subsection{SNInfo and ChannelInfo}

ChannelInfo keeps for all AirFrames on the channel. ChannelInfo does not differ between \textit{signal} and \textit{noise}. \h{\bp} is able to
add and remove references to single AirFrames.
ChannelInfo is able to record the whole channel over time from a start to a stop signal and can return a vector of Signals (references) that intersect with a given time interval.\\
SNInfo is created by \h{\bp} when a packet arrives to collect all signals from the channel that intersect with the reception time interval.

\begin{figure}[h]
 \centering
 \includegraphics[width=340pt]{modelling/ChannelInfo_members.png}
 \caption{channel details}
 \label{fig: channel details}
\end{figure}



\subsection{Decider}

The decider has two tasks:
\begin{enumerate}
	\item It decides whether we are able to receive a certain packet by evaluting
	the SNInfo for the packets preamble time interval, otherwise the packet will 	be considered noise
	\item When a packet has been received as a \textit{signal} the Decider 	returns a DeciderResult for that packet, that only contains correct/not correct 	by default.
\end{enumerate}


\begin{figure}[h]
 \centering
 \includegraphics[width=340pt]{modelling/DeciderModule_members.png}
 \caption{Decider interface}
 \label{fig: Decider interface}
\end{figure}


\subsection{AirFrame}


\begin{figure}[h]
 \centering
 \includegraphics[width=300pt]{modelling/AirFrame_members.png}
 \caption{member arrangement in AirFrame and Signal}
 \label{fig: member AirFrame}
\end{figure}

\subsection{receiving an AirFrame}


\begin{figure}[h]
 \centering
 \includegraphics[width=340pt]{modelling/onAirFrame.png}
 \caption{receiving process}
 \label{fig: receiving process}
\end{figure}

\begin{figure}[h]
 \centering
 \includegraphics[width=340pt]{modelling/receive_detail.png}
 \caption{receive detail}
 \label{fig: receive detail}
\end{figure}

\subsection{receiving a MacPkt}


\begin{figure}[h]
 \centering
 \includegraphics[width=300pt]{modelling/onMACPkt.png}
 \caption{sending process}
 \label{fig: sending process}
\end{figure}


%\subsection{provide status information to MAC}

%Passively provided information\req{provpassive}: \h{\bm} is equipped with a reference to \h{\bp} in order to obtain information
%about channelstate\req{channelstate} and current mode\req{currentmode} by
%simple method calls. \\
%Actively provided information\req{provactive}: A cMessage of the kind TX\_OVER
%is sent to MAC-Layer when a sending transmission is over\req{txover}, \saf{sending process}.

%\subsection{switch states}

%Furthermore the MAC-Layer is able to set the current mode (RX, TX, SLEEP)\req{switchmode} of the Phy-Layer by a simple method call.
%\h{\bp} will change to switching state and schedule itself the appropriate timer for the switching interval\req{switchtimes}, \saf{mode state machine}.


%\subsection{send packets}

%Since \h{\bm} has a reference to \h{\bp} it can obtain information about the mode the radio is is currently in\req{sendPreqMode}, it is not already sending to the channel on its own\req{sendPreqSending} and the channel is idle\req{sendPreqIdle} via method calls, \saf{BasePhyLayer interface}.

%The class MacToPhyControlInfo is designed as the container for control info\req{packetFromMac} the MAC-Layer
%wants to attach to the packet given down to Phy-Layer for sending.
%The packet itself is handed down as a MacPkt via OMNeT-channel. 

%\begin{figure}[h]
% \centering
% \includegraphics[width=340pt]{modelling/MacToPhyCtrlInfo_members.png}
% \caption{MacToPhyControlInfo interface}
% \label{fig: MacToPhyCtrlInfo interface}
%\end{figure}













